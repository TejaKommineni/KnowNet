%%% -*-LaTeX-*-

\chapter{Conclusion} \label{chap:conclusion}

Section ~\ref{contributions} remarked the following thesis statement:

''By combining clustering and other data mining techniques we can produce a view of host behaviors that help network administrators to understand the behavior of the networks. This system complements existing network analysis tools to ease network management.''

We have found conclusive evidence for this statement by evaluating the university network data and the labeled data obtained from Center for Applied Internet Data Analysis (CAIDA). We analyzed hosts and studied the behavior of groups of hosts. This research of grouping hosts based on their behaviors is one of a kind and was conducive in comprehending the behavior of networks. The cross-validation of university data and the high accuracy obtained in classifying labeled data confirm this fact. This system with the help of existing tools helped us in making recommendations to the network administrators about different management activities. While evaluating the university network data, we found hosts exhibiting seven unique behaviors. We had hosts behaving as clients, servers, exchanging normal traffic and scanners. Few of these behaviors extracted by our system were in-line with the expected behaviors by the network admins and others helped them better understand the network. We also presented a web-based assisting tool that helps in visualizing these behaviors over time periods and look into historic behaviors that this network exhibited.

The primary contributions of this work are threefold. First, an 8D feature vector has been defined and used to characterize the host behavior accurately. Our feature set supports high volumes of traffic and doesn't consider payload. Also, the feature vector is of low dimension (8D) signifying the richer information each feature contains.
Second, the grouping is based on the K-means approach: this is an unsupervised technique which doesn't require ground truth knowledge about the data set. It doesn't start with any initial assumptions with respect to the number of clusters and is adaptive in nature by adjusting itself to new classes of traffic previously unobserved. Third, the systems performance and functionality are tested on a real dataset collected at the University routers and the labeled dataset obtained from CAIDA. Cross-validation with port-based approaches and accuracy on characterizing and classifying the labeled data respectively reflect the potentiality of our approach and the relevance of our procedure.
Hence, our contribution shows that the 8D feature vector, that captures the behavior of a host, combined with K-means clustering technique gives us classification of host behaviors in an unsupervised manner.

\section{Future Directions}

\begin{enumerate}
	\item Investigate and build applications using the results of our system. Our system groups hosts exhibiting similar behaviors and defines each group by certain characteristics such as the cluster sizes, cluster centers etc.. Using this information about each group we can build various applications such as anomaly detectors, dynamic firewall rule generators or bandwidth predictors which can become further handy to network administrators in their day-to-day activities.
	
	\item In our work, we define host as any unique IP address that has in-bound or outbound traffic. But, network users needn't always have a unique IP address. Each user can be communicating in a network with multiple IP addresses. This could happen because of a malicious user trying to hide his identity by using different IP addresses or because of dynamic allotment by ISP provider. In that direction, it would be useful to group hosts at a user level instead of using IP address as a basis. Even in this approach most of our techniques should work well. 
	
	\item When analyzed with labeled data our system had higher accuracy in scenario1 and scenario3 but when it comes to detecting chinese hosts or multi-network scenarios our rate of detection is a bit low. We could further improve the accuracy by tuning our system. Few pointers to explore here are to evaluate different normalizing techniques.
	
	\item When multiple networks are fused together we see a high  traffic in such setting. Exploring streaming algorithms \cite{guha2016clustering} which leave a lesser memory footprint while capturing various characteristics of the data or  concurrent programming to speed up the individual components of our proposed procedure to improve performance is advised.
\end{enumerate}


\section{Parting Thoughts}
Today's system administrator has to navigate through a set of tools to manage networks. It is a complex job to do and is critical for maintaining network performance. In our work we have built a tool that complements the existing tools to ease the network admins burden and our observation is that further research efforts should be invested in trying to provide a holistic view of the network through a single tool to the administrator thus requiring less effort. It would be interesting to also embed intelligence into these tools that can capture network admins experiences and decisions. These efforts should also take into consideration the changing characteristics of the network and provide solutions that adapt with time. We believe there is a lot of potential for ML/DM algorithms in doing so.
